\newglossaryentry{S}{
        name={$S$},
        description={множество состояний} 
}
\newglossaryentry{X}{
        name={$X$},
        description={множество входных символов} 
}
\newglossaryentry{Y}{
        name={$Y$},
        description={множество выходных символов} 
}
\newglossaryentry{d}{
        name={$\delta$},
        description={функция перехода} 
}
\newglossaryentry{l}{
        name={$\lambda$},
        description={} 
}
\newglossaryentry{M}{
        name={$M$},
        description={конечный автомат} 
}
\newglossaryentry{Minv}{
        name={$M^{'}$},
        description={обратный к $M$ конечный автомат} 
}
\newglossaryentry{XN}{
        name={$X^{n}$},
        description={множество входных последовательностей длины $n$} 
}
\newglossaryentry{YN}{
        name={$Y^{n}$},
        description={множество выходных последовательностей длины $n$} 
}
\newglossaryentry{xN}{
        name={$x^{n}$},
        description={выходная последовательность длины $n$} 
}
\newglossaryentry{yN}{
        name={$y^{n}$},
        description={выходная последовательность длины $n$} 
}
\newglossaryentry{Lambda}{
        name={$\Lambda$},
        description={пустой символ} 
}
\newglossaryentry{N}{
        name={$N$},
        description={порядок набора состояний} 
}
\newglossaryentry{YNsi}{
        name={$Y^{n}(s_{i})$},
        description={} 
}
\newglossaryentry{Siyn}{
        name={$Y^{n}(s_{i})$},
        description={} 
}
\newglossaryentry{Xnsiyn}{
        name={$Y^{n}(s_{i})$},
        description={} 
}
\newglossaryentry{Sfsiyn}{
        name={$Y^{n}(s_{i})$},
        description={} 
}
\newglossaryentry{Fnsi}{
        name={$Y^{n}(s_{i})$},
        description={} 
}
\newglossaryentry{emptyset}{
        name={$\emptyset$},
        description={пустое множество} 
}
\newglossaryentry{L}{
        name={$L$},
        description={задержка обратимости} 
}
\newglossaryentry{invL}{
        name={$INV \#L$},
        description={обратимость с задержкой $L$} 
}
\newglossaryentry{IL}{
        name={$IL$},
        description={устойчивость к потере информации} 
}
\newglossaryentry{ILF}{
        name={$ILF$},
        description={устойчивость к потере информации конечного порядка} 
}
\newglossaryentry{On}{
        name={$O_{n}$},
        description={выходная $n$-эквивалентность} 
}
\newglossaryentry{Pin}{
        name={$\pi_{n}$},
        description={множество разбиений состояний по $O_{n}$} 
}
\newglossaryentry{OMR}{
        name={$OMR$},
        description={} 
}
\newglossaryentry{FIM}{
        name={$FIM$},
        description={конечная входная память} 
}
\newglossaryentry{FOM}{
        name={$FOM$},
        description={конечная выходная память} 
}
\newglossaryentry{f}{
        name={$f$},
        description={функция выходов} 
}
\newglossaryentry{mu}{
        name={$\mu$},
        description={входная или выходная память} 
}
\newglossaryentry{LSC}{
        name={$LSC$},
        description={линейная схема} 
}
\newglossaryentry{GDHD}{
        name={$G(D); H(D)$},
        description={фнукции преобразования матриц} 
}
\newglossaryentry{ID}{
        name={$\underline{I}(D)$},
        description={преобразованный входной вектор} 
}
\newglossaryentry{TD}{
        name={$\underline{T}(D)$},
        description={преобразованный выходной вектор} 
}
\newglossaryentry{ABCE}{
        name={$A, B, C, E$},
        description={структурные матрицы $LSC$} 
}
\newacronym{fsm}{КА}{конечный автомат}
\newacronym{gcd}{НОД}{наибольший общий делитель}
\newacronym{lcm}{НОК}{наименьшее общее кратное}



\newglossaryentry{maths}
{
    name=mathematics,
    description={Mathematics is what mathematicians do}
}
 
\newglossaryentry{latex}
{
    name=latex,
    description={Is a mark up language specially suited for 
scientific documents}
} 
 
 
\newglossaryentry{formula}
{
    name=formula,
    description={A mathematical expression}
}

\newglossaryentry{formula2}
{
    name=Формула,
    description={Выражение}
}